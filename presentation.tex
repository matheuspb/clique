\documentclass[]{beamer}
\usepackage[brazilian]{babel}
\usepackage[utf8]{inputenc}
\usepackage{lmodern}
\usepackage{tikz}

\tikzstyle{every node}=[draw,shape=circle,align=center,anchor=north]
\tikzstyle{v}=[scale=0.8]
\tikzstyle{vf}=[fill, scale=0.5]
\tikzstyle{tl}=[very thick]

\begin{document}

\title{Prova da \textit{NP}-completude do problema do \textit{Clique}}

\author{
	Matheus Silva P. Bittencourt~\inst{1} \and
	Filipe de Oliveira Borba~\inst{1}
}

\institute{\inst{1} Universidade Federal de Santa Catarina}

\date{}

\frame{\titlepage}

\begin{frame}
	\frametitle{O problema do \textit{Clique}}

	\begin{figure}
		\begin{tikzpicture}
			\node [vf] (v1) at (0, 0) {};
			\node [vf] (v2) at (2, 0) {};
			\node [vf] (v3) at (1, 1) {};
			\node [vf] (v4) at (3, 1) {};
			\node [vf] (v5) at (0, 2) {};
			\node [vf] (v6) at (2, 2) {};

			\draw (v1) -- (v2);
			\draw (v1) -- (v3);
			\draw (v1) -- (v5);
			\draw (v5) -- (v6);

			\draw [tl] (v2) -- (v3);
			\draw [tl] (v2) -- (v4);
			\draw [tl] (v2) -- (v6);
			\draw [tl] (v3) -- (v4);
			\draw [tl] (v3) -- (v6);
			\draw [tl] (v6) -- (v4);
		\end{tikzpicture}
	\end{figure}

	\begin{itemize}[<+->]
		\item Um \textit{clique}, em um grafo não orientado, é um subgrafo
			completo, ou seja, um subgrafo tal que cada vértice está conectado
			a todos os outros do subgrafo.
		\item Um \textit{k-clique} é um \textit{clique} com $k$ vértices.
		\item $CLIQUE = \{ \langle G,k \rangle \mid G \text{ possui um
			\textit{k-clique}} \}$
		\item $CLIQUE \in NP$, pois possui um verificador em tempo polinomial.
	\end{itemize}

\end{frame}

\begin{frame}
	\frametitle{Ideia da prova}

	\begin{itemize}[<+->]
		\item Reduzir o \textit{3SAT} ao problema do \textit{clique}.
		\item $3SAT = \{\langle \phi \rangle \mid \phi \text{ é uma
			fórmula-FNC3}\footnote{forma normal conjuntiva 3}\text{,
			satisfazível}\}$
		\item De acordo com o teorema de Cook-Levin, $3SAT$ é
			\textit{NP}-completo.
		\item Se $3SAT \leq_p CLIQUE$, então $CLIQUE$ é \textit{NP}-completo.
		\item Converter uma fórmula $\phi$, para um grafo $G$, de tal forma que
			$\phi \in 3SAT \iff G \in CLIQUE$.
		\item Tal conversão pode ser feita em tempo polinomial.
	\end{itemize}

\end{frame}

\begin{frame}
	\frametitle{Redução}

	Seja $\phi$ uma fórmula com $k$ clausulas na forma:
	$$\phi = (a_1 \vee b_1 \vee c_1) \wedge (a_2 \vee b_2 \vee c_2) \wedge ...
	\wedge (a_k \vee b_k \vee c_k)$$
	a redução consiste em gerar $\langle G,k \rangle$ a partir de $\phi$.

	\begin{itemize}[<+->] \pause
		\item Para cada clausula, criar 3 vértices correspondentes as variáveis
			da clausula, formando as triplas $t_1,\dots,t_k$
		\item Conectar todos os vértices, com exceção de 2 casos:
		\item Vértices que estão na mesma tripla (clausula).
		\item Vértices de variáveis que se contradizem, por exemplo,
			$x_i$ e $\overline{x_i}$.
	\end{itemize}
\end{frame}

\begin{frame}
	\frametitle{Exemplo}

	$$\phi = (x_1 \vee x_1 \vee x_2) \wedge (\overline{x_1} \vee \overline{x_2}
	\vee \overline{x_2}) \wedge (\overline{x_1} \vee x_2 \vee x_2)$$

	\begin{figure}
		\begin{tikzpicture}
			\node<1-> [v] (a1) at (0, 0) {$x_1$};
			\node<1-> [v] (b1) at (0, 1) {$x_1$};
			\node<1-> [v] (c1) at (0, 2) {$x_2$};

			\node<1-> [v] (a2) at (1.2, 3.2) {$\overline{x_1}$};
			\node<1-> [v] (b2) at (2.4, 3.2) {$\overline{x_2}$};
			\node<1-> [v] (c2) at (3.6, 3.2) {$\overline{x_2}$};

			\node<1-> [v] (a3) at (4.8, 2) {$\overline{x_1}$};
			\node<1-> [v] (b3) at (4.8, 1) {$x_2$};
			\node<1-> [v] (c3) at (4.8, 0) {$x_2$};

			\draw<2-> (a1) -- (b2);
			\draw<2-> (a1) -- (c2);
			\draw<2-> (a1) -- (b3);
			\draw<2-> (a1) -- (c3);

			\draw<3-> (b1) -- (b2);
			\draw<3-> (b1) -- (c2);
			\draw<3-> (b1) -- (b3);
			\draw<3-> (b1) -- (c3);

			\draw<3-> (c1) -- (a2);
			\draw<3-> (c1) -- (a3);
			\draw<3-> (c1) -- (b3);
			\draw<3-> (c1) -- (c3);

			\draw<3-> (a2) -- (a3);
			\draw<3-> (a2) -- (b3);
			\draw<3-> (a2) -- (c3);

			\draw<3-> (b2) -- (a3);
			\draw<3-> (c2) -- (a3);

			\draw<4-> [tl] (c1) -- (a2);
			\draw<4-> [tl] (a2) -- (c3);
			\draw<4-> [tl] (c3) -- (c1);
		\end{tikzpicture}
	\end{figure}

\end{frame}

\begin{frame}
	\frametitle{Prova}

	\center{$\phi \in 3SAT \implies G \in CLIQUE$}

	\begin{itemize}[<+->]
		\item Suponha que $\phi$ possui valores satisfatórios, ou seja, cada
			clausula possui uma variável verdadeira.
		\item Em cada tripla de $G$ selecione um vértice correspondente a uma
			variável verdadeira da clausula.
		\item Os vértices selecionados formam um \textit{k-clique}.
	\end{itemize}

\end{frame}

\begin{frame}
	\frametitle{Prova}

	\center{$G \in CLIQUE \implies \phi \in 3SAT$}

	\begin{itemize}[<+->]
		\item Suponha que $G$ tem um \textit{k-clique}.
		\item Cada tripla possui um vértice no \textit{clique}.
		\item Para cada vértice no \textit{clique}, faça com que sua variável
			correspondente em $\phi$ seja verdadeira.
		\item $\phi$ é satisfeito.
	\end{itemize}

\end{frame}

\begin{frame}
	\frametitle{Conclusão}

	$$\phi \in 3SAT \iff G \in CLIQUE$$

	\begin{itemize}[<+->]
		\item Dada uma fórmula $\phi$ podemos convertê-la, em tempo polinomial,
			a um grafo $G$, de tal forma que, achar um \textit{k-clique} no
			grafo $G$ implica em achar os valores que satisfazem $\phi$.
		\item Um algoritmo em tempo polinomial que resolva o problema do
			\textit{clique}, pode ser utilizado para resolver o \textit{3SAT},
			em tempo polinomial.
		\item O problema do \textit{clique} é no mínimo tão difícil quanto o
			\textit{3SAT}.
		\item $CLIQUE$ é \textit{NP}-Completo.
	\end{itemize}

\end{frame}

\end{document}
